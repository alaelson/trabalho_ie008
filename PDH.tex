\documentclass[a4paper,12pt]{article}

\usepackage{amssymb}
\usepackage{graphicx}
\usepackage[utf8]{inputenc}
\usepackage[normalem]{ulem}
\usepackage[brazil]{babel}
\usepackage{amsmath}
\usepackage{wrapfig}
\usepackage{color}
\usepackage{tikz}
\usepackage{subfigure}
\usepackage{latexsym}
\usepackage{setspace}  %Alteração dos espaçamento entre linhas
\usepackage{blindtext}
\usepackage{epstopdf}

\usepackage{tikz}
\usetikzlibrary{chains}

\usepackage{epsfig,enumerate,subfigure,multirow}

\usepackage{wrapfig}

\setlength{\hoffset}{-1cm}
\setlength{\voffset}{-1cm}
\setlength{\textheight}{24cm}
\setlength{\textwidth}{16cm}

\doublespacing 


\begin{document}

\section{Introdução}

De um ponto de vista geral, os sistemas de comunicações podem ser organizados em um hierarquia de camadas, tal como observado na figura \ref{fig1}, na qual cada segmentação permite o encapsulamento dos dados e assegura uma transmissão confiável e transparente. Neste contexto, o cliente insere tráfego de dados, vídeo e voz, utilizando a camada física como caminho interligando diversos nós na rede. Cada camada, encapsula informação útil a ser enviada, e para que a comunicação seja eficaz, é necessário que os protocolos de interfaceamento estejam bem definidos.

Na camada de transporte, os protocolos mais utilizados atualmente em redes ópticas são: Rede Óptica Síncrona- SONET (do inglês, {\it Synchonous Optical Network}), Hierarquia Digital Síncrona -SDH(do inglês, {\it Synchonous Digital Hierarchy}), IP e ATM. SONET e SDH surgiram quase que simultaneamente por volta da década de 80, na América do Norte e Europa, respectivamente, devido a necessidade das companhias telefônicas em conectar suas redes e são utilizadas amplamente até hoje.

\begin{figure}[!htb]


\begin{center}

\begin{tikzpicture}[
	scale=0.75,
	start chain=1 going below, 
	start chain=2 going right,
	node distance=1mm,
	desc/.style={
		scale=0.75,
		on chain=2,
		rectangle,
		rounded corners,
		draw=black, 
		very thick,
		text centered,
		text width=8cm,
		minimum height=12mm,
		fill=blue!40
		},
	it/.style={
		fill=blue!10
	},
	level/.style={
		scale=0.75,
		on chain=1,
		minimum height=12mm,
		text width=2cm,
		text centered
	},
	every node/.style={font=\rm}
]

% Levels
\node [level] (Level 5) {Camada 5};
\node [level] (Level 4) {Camada 4};
\node [level] (Level 3) {Camada 3};
\node [level] (Level 2) {Camada 2};
\node [level] (Level 1) {Camada 1};


% Descriptions
\chainin (Level 5); % Start right of Level 5
% IT levels
\node [desc, it] (Archives) {Aplicação};
\node [desc, it, continue chain=going below] (ERP) {Redes};
\node [desc] {Enlace};
\node [desc] {Transporte};
\node [desc] {Física};

\end{tikzpicture}
\caption{Hierarquia de camadas nos sistemas de comunicações \label{fig1}}
\end{center}
\end{figure}




\section{Histórico/PDH}

Anteriormente a popularização dos padrões SONET/SDH, a infraestrutura de transporte baseava-se na Hierarquia Digital Plesiócrona-PDH ({do inglês, \it Plesiochnous Digital Hierarchy}), datada entre o início dos anos 70 até o final da década de 90. O PDH é um protocolo de comunicação tradicionalmente desenvolvido para telefonia e que permite o envio de vários sinais sobre o mesmo meio, seja cabo coaxial ou fibra óptica, utilizando técnicas de multiplexação no tempo e dados em formato digital. O termo plesiocróno se refere ao fato de que as diferentes terminações da rede operam com relógio quase que sincronizados. Como o foco das pesquisas naquele momento estava na comunicação de voz, demandava-se uma largura de banda de 8 kHz para comunicação bilateral, o que a 8 bits por amostra implicava na digitalização a uma taxa de 64 kb/s. Este valor tornou-se padrão e as transmissões de valores superiores foram niveladas em múltiplos desse, conforme destacado na tabela \ref{tab1}, na qual observa-se as taxas padronizadas na hierarquia. Porém, PDH sofria de muitos problemas, o que impulsionou pesquisas de um novo padrão de multiplexação que se consagrou na forma do SONET/SDH. As principais limitações do PDH, naquele momento eram:

\begin{table}
\centering
\begin{tabular}{c|c|c|c}
\hline
{\bf Nível} & {\bf América do Norte} & {\bf Europa} & {\bf Japão} \\ \hline
0 & 0.064 Mb/s & 0.064 Mb/s & 0.064 Mb/s \\
1 & 1.544 Mb/s & 2.048 Mb/s & 1.544 Mb/s \\
2 & 6.321 Mb/s & 8.448 Mb/s & 6.321 Mb/s \\
3 & 44.736 Mb/s & 34.368 Mb/s & 32.064 Mb/s \\
4 & 139.264 Mb/s & 139.264 Mb/s & 97.728 Mb/s \\ \hline
\end{tabular}
\caption{Taxas de transmissão para PDH \cite{ramaswami} \label{tab1}}
\end{table}

\begin{enumerate}


\item {\bf Preenchimento de pacotes:} como cada terminal opera com um relógio individual ligeiramente próximo dos demais, pode ocorrer diferenças significativas entre dois relógios. Assim, antes de realizar a intercalação é preciso manter os enlaces no mesmo ritmo, o que é feito adicionando bit sem informação (bits vazios) que servem para  compensar diferenças de relógio, bits estes descartados na demultiplexação.

\item {\bf Extração de níveis inferiores:} outra limitação era a dificuldade de extrair um sinal de ordem inferior de um de ordem superior, sem demultiplexar os passos intermediários, ou seja, para extrair o sinal de um enlace em determinado ponto era necessário uma demultiplexação sucessiva até o ponto desejado seguida de uma remultiplexação para reenvio do sinal.
\item {\bf Gerenciamento:} PDH não incorporava informações de gerenciamento da rede, monitoramento e identificação de conectividade, relatório de falhas ou um caminho de comunicação de dados para transporte dessas informações. Isso ocorria devido a limitação de taxa transportada que não permitia alocação bits para essas funções.
\item {\bf Compatibilidade:} a falta de descrição do formato de transmissão no meio implicava no desenvolvimento de diferentes formatos de codificação e interfaceamento, levando a um problema de incompatibilidade entre equipamentos de diferentes desenvolvedores.
\item {\bf Disponibilidade da rede:} o tempo de restauração de falhas com o protocolo PDH variava entre segundos até minutos, valor muito superior aos protocolos sucessores.

\end{enumerate}

\begin{thebibliography}{10}


 \bibitem{ramaswami} RAMASWAMI, Rajiv; SIVARAJAN, Kumar; SASAKI, Galen. {\bf Optical networks: a practical perspective}. Morgan Kaufmann, 2009.

\end{thebibliography}

\end{document}