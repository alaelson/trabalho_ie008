%%
%% ********** Resumo
%%
\newpage \thispagestyle{plain} 
\vspace{1.5cm}
\begin{center}
{\huge{\textbf{Resumo}}}
\end{center}
\vspace{0.5cm}

O aumento do n�mero de clientes dos servi�os de comunica��es combinado com a mudan�a do perfil do usu�rio, que demanda maiores taxas de transmiss�o, servi�os de alta qualidade e mobilidade de acesso, tem levado ao crescimento acelerado do tr�fego de dados nas redes  de comunica��es. Faz-se, ent�o, necess�rio o cont�nuo desenvolvimento das
tecnologias de comunica��es �pticas, de seus protocolos de transporte, controle eficiente da
rede, al�m de t�cnicas de resili�ncia e sobreviv�ncia em caso de falhas. Neste trabalho s�o apresentadas alguns dos protocolos de redes �pticas de transporte, que tem a fun��o de encaminhar os dados e transmiti-los sobre o meio f�sico, e dos planos de controle que atuam sobre o gerenciamente destas. S�o apresentadas as principais tecnologias de transporte �ptico: Hierarquia Digital Plesi�crona (PDH), Rede �tica S�ncrona (SONET), Hierarquia Digital S�ncrona (SDH) e Rede de Transporte �tica (OTN). Adicionalmente, s�o detalhados os protocolos de controle como GMPLS (Generalized Multiprotocol Label Switching) e as Redes Definidas por software (SDN). Um estudo de caso com a aplica��o dos conceitos envolvidos � relatado, ilustrando a aplicabilidade  dos t�picos discutidos.

\vspace{1.5ex}

\textbf{Palavras-chave}: Protocolos de Transporte, Rede �tica, Rede transporte �tica, PDH, SONET/SDH, GMPLS, SDN.