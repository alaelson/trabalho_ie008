%%
%% ********** Resumo
%%
\newpage \thispagestyle{plain} 
\vspace{1.5cm}
\begin{center}
{\huge{\textbf{Resumo}}}
\end{center}
\vspace{0.5cm}

Este trabalho apresenta uma plataforma para o gerenciamento e aprovisionamento de caminhos �ticos que faz uso de plano de controle GMPLS (\textit{Generalized Multiprotocol Label Switching}) em uma rede de transporte �tica que atende � recomenda��o ITU-T G.709. Para isto, um novo algoritmo foi desenvolvido e implementado para o controle de admiss�o de conex�es baseado em restri��es de camada f�sica, que suporta o protocolo RSVP-TE (\textit{Resource Reservation Protocol -- Traffic Engineering}) para utilizar as informa��es sobre degrada��o do sinal �tico, incorporando somente tr�s novos indicativos nas mensagens \textit{NOTIFY}. Essas adi��es referem-se a alarmes disparados devido a problemas de transmiss�o que podem ocorrer em sinais �ticos ou el�tricos. O monitoramento � feito analisando-se a pot�ncia �tica transmitida e recebida pelos terminais. Al�m disso, a taxa de erro de bits (BER) � monitorada ap�s a execu��o do algoritmo de corre��o de erros (FEC) que � implementado no n� �tico. Para validar o algoritmo, utilizou-se um enlace experimental ponto a ponto de $10~Gbps$ com sinaliza��o fora da banda do projeto Kyatera, que conecta a Universidade Federal do ABC (UFABC), campus Santo Andr�, � Universidade Estadual de Campinas (UNICAMP).

\vspace{1.5ex}

\textbf{Palavras-chave}: Rede �tica, rede transporte �tica, plano de controle GMPLS, multiplexa��o por divis�o de comprimento de onda, aprovisionamento, monitoramento, caminhos �ticos, ITU-T G.709.