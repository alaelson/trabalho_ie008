%%
%% ********** Abstract
%%
\newpage \thispagestyle{plain} 
\vspace{1.5cm}
\begin{center}
{\huge{\textbf{Abstract}}}
\end{center}
\vspace{0.5cm}

This work presents a platform to manage and to provision lightpaths using control plane of the Generalized Multi-protocol Label Switch (GMPLS) in optical transport networks (OTN, ITU-T G.709). In order to make control of the optical connection requests, a new impairment-aware algorithm is proposed, which supports the Resource Reservation Protocol -- Traffic Engineering (RSVP-TE) of the GMPLS signaling by deployment of the impairment information. In fact, it adds three new indications in the NOTIFY messages. These indications refer to alarms that can be triggered caused by transmission problems, which may arise either in the optical or electrical signals. The monitoring is performed checking the transmitted and received optical power on the optical nodes. Moreover, the bit error rate (BER) is monitored after carrying out forward error correction algorithm (FEC) in the optical node. In order to validate the proposed algorithm, it was deployed a high speed experimental optical link ($10~Gbps$) of the KyaTera Project which connects two universities, Federal University of ABC (UFABC) and State University of Campinas (UNICAMP).

\vspace{1.5ex}

\textbf{Palavras-chave}: Optical network, optical transport network, GMPLS control plane, wavelength division multiplexing, provisioning, monitoring, lightpath, ITU-T G.709.
