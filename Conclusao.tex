\chapter{Conclus�o}
\label{cap:conclusion}

Este trabalho teve como intuito abordar as tecnologias de transporte em redes �pticas e apresentar conceitos relacionados ao seu gerenciamento. � poss�vel perceber que o aumento do tr�fego de dados impulsiona a gera��o de novas tecnologias de transporte para redes de telecomunica��es, em especial sobre a infraestrutura de fibras �pticas. Por volta da d�cada de 1960, os sistemas de telecomunica��es tinham como intuito atender simplesmente as necessidades dos canais telef�nicos atrav�s de PDH, mas o aumento da demanda por servi�os aliados as dificuldades t�cnicas de PDH conduziram ao desenvolvimento de novas tecnologias de transporte como SONET e SDH, ambos voltados a aplica��o sobre fibras �pticas. Com o avan�o e a maturidade das tecnologias de comunica��es �pticas juntamente com o processo de converg�ncia dos servi�os de telecomunica��es, passou-se a ser necess�rio a evolu��o ainda maior dos sistemas de transporte bem como a integra��o entre as redes, surgindo ent�o GMPLS e OTN. Atualmente percebe-se um grande interesse sobre o conceito de SDN, que traz um novo paradigma as redes de transporte por meio da possibilidade de flexibilidade de rearranjo da rede, dentre outros fatores. 

Conceitos de flexibilidade, escalabilidade, programabilidade e redu��o de custos em paralelo com a sempre crescente demanda de velocidade e qualidade dos servi�os de telecomunica��es se apresentam como desafios inerentes as tecnologias de transporte e gerenciamento das redes no cen�rio atual, onde OTN e SDN demonstram ser as op��es escolhidas por parte dos fornecedores de servi�os para solucionar as necessidades sempre utilizando como meio f�sico primordial as fibras �pticas.  