\chapter{Introdu��o}
\label{cap:introduction}

\section{Objetivo}

O objetivo deste trabalho � fazer uma breve explana��o sobre as principais tecnologias de transporte �ptico, explicitando os aspectos t�cnicos de cada uma.

\section{Introdu��o}

De um ponto de vista geral, os sistemas de comunica��es podem ser organizados em um hierarquia de camadas, tal como observado na figura \ref{fig:Camadas}, na qual cada segmenta��o permite o encapsulamento dos dados e assegura uma transmiss�o confi�vel e transparente. Neste contexto, o cliente insere tr�fego de dados, v�deo e voz, utilizando a camada f�sica como caminho interligando diversos n�s na rede. Cada camada, encapsula informa��o �til a ser enviada, e para que a comunica��o seja eficaz, � necess�rio que os protocolos de interfaceamento estejam bem definidos.

Na camada de transporte, os protocolos mais utilizados atualmente em redes �pticas s�o: Rede �ptica S�ncrona- SONET (do ingl�s, {\it Synchonous Optical Network}), Hierarquia Digital S�ncrona - SDH(do ingl�s, {\it Synchonous Digital Hierarchy}), IP e ATM. SONET e SDH surgiram quase que simultaneamente por volta da d�cada de 80, na Am�rica do Norte e Europa, respectivamente, devido a necessidade das companhias telef�nicas em conectar suas redes e s�o utilizadas amplamente at� hoje.

\begin{figure}[!htb]

\begin{center}

\begin{tikzpicture}[
	scale=0.75,
	start chain=1 going below, 
	start chain=2 going right,
	node distance=1mm,
	desc/.style={
		scale=0.75,
		on chain=2,
		rectangle,
		rounded corners,
		draw=black, 
		very thick,
		text centered,
		text width=8cm,
		minimum height=12mm,
		fill=blue!40
		},
	it/.style={
		fill=blue!10
	},
	level/.style={
		scale=0.75,
		on chain=1,
		minimum height=12mm,
		text width=2cm,
		text centered
	},
	every node/.style={font=\rm}
]

% Levels
\node [level] (Level 5) {Camada 5};
\node [level] (Level 4) {Camada 4};
\node [level] (Level 3) {Camada 3};
\node [level] (Level 2) {Camada 2};
\node [level] (Level 1) {Camada 1};


% Descriptions
\chainin (Level 5); % Start right of Level 5
% IT levels
\node [desc, it] (Archives) {Aplica��o};
\node [desc, it, continue chain=going below] (ERP) {Redes};
\node [desc] {Enlace};
\node [desc] {Transporte};
\node [desc] {F�sica};

\end{tikzpicture}
\caption{Hierarquia de camadas nos sistemas de comunica��es \label{fig:Camadas}}
\end{center}
\end{figure}