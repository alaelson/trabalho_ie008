
A tecnologia de transporte PDH, que foi introduzida para atender a demanda de multiplexa��o de canais de voz, apresentava uma s�rie de limita��es tecnol�gicas e operacionais, o que fez com que na d�cada de 1980 fornecedores de servi�os de telecomunica��es e fabricantes de equipamentos buscassem novos padr�es de transmiss�o e multiplexa��o. Em 1984 a associa��o americana chamada \textit{Exchange Carriers Standards Association} (ECSA) iniciou trabalhos com a finalidade de desenvolvimento de um padr�o para o \textit{American National Standards Institute} (ANSI) e no ano de 1985 foi criado o padr�o de transporte em redes de telecomunica��es chamado de \textit{Synchronous Optical Network} (SONET), adotado em pa�ses como Estados Unidos, Canada e Jap�o. O \textit{International Telegraph and Telephone Consultative Committee} (CCITT), atualmente conhecido como \textit{International Telecommunication Union Telecommunication Standardization Sector} (ITU-T), interessou-se pelo padr�o ANSI e prop�s mudan�as para a acomoda��o de ambas as hierarquias de taxas de transmiss�o americanas e europeias e finalmente em 1988 alcan�ou-se um acordo e o �rg�o CCITT definiu o padr�o conhecido como \textit{Synchronous Digital Hierarchy} (SDH), amplamente adotado na Europa e em pa�ses como o Brasil. Devido ao fato de que SDH foi desenvolvido a partir de SONET, ambos guardam muitos conceitos tecnol�gicos semelhantes, onde as diferen�as se situam em certas denomina��es de par�metros e nas velocidades das hierarquias.
